%implementing document formatting:
\input{preamble.tex}
%Vectors
\renewcommand{\vec}[1]{\boldsymbol{\mathbf{#1}}}
\begin{document}
\renewcommand\chaptername{KAPITEL}
\renewcommand\contentsname{Indhold}
\renewcommand\figurename{Figur}
\renewcommand\tablename{Tabel}

\section*{Supervisor meeting\\ \small Wednesday, 19th of October 2016}

\subsection{Paper Structure}
\begin{itemize}
  \item[-] The proper names for the controllers could be Angular Velocity controller, Angle controller, Translational Velocity Controller.
  \item[-] The name "method" is fine. It represents all the methods we use to get to our results and how we use them.
  \item[-] The degree of explanation in the paper depends on how good the sources are and what we assume our peers to know. It is important to take into account that they have not worked on the project as much as us.
  \item[-] Analysis of mistakes or weird behavior go in the results section.
  \item[-] The discussion is taking a step back and stating what we got out of the work. A discussion in general. Not talk about improvements in the discussion, if we have improvements, the paper is not ready to publish.
  \item[-] Conclusion may not be needed as it repeats things already said.
  \item[-] Last part of the paper structure will emerge as we write.
\end{itemize}

\subsection{Setup to Estimate the Moments of Inertia}
\begin{itemize}
  \item[-] We need to find the center of mass.
  \item[-] We need to know the applied torque to the system and how it reacts.
  \item[-] Procedure: Swing the drone with two wires for calculating the moment of inertia around x and y axes.
  \item[-] To calculate the moment of inertia around z axis, we need to swing it so it turns around the z axis.
  \item[-] Initial estimation could be an analytical calculation as we know the mass and we can assume a ring-shape mass distribution. This could help us check if our results from the experiment are reasonable.
\end{itemize}

\subsection{Next Supervisor Meeting}
Monday, 26th of October at 11:15

\end{document}