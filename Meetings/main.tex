%implementing document formatting:
\input{preamble.tex}
%Vectors
\renewcommand{\vec}[1]{\boldsymbol{\mathbf{#1}}}
\begin{document}
\renewcommand\chaptername{KAPITEL}
\renewcommand\contentsname{Indhold}
\renewcommand\figurename{Figur}
\renewcommand\tablename{Tabel}

\section*{Supervisor meeting\\ \small Wednesday, 5th of October 2016}

\subsection{The examination}
\begin{itemize}
  \item[-] The poster is normally presented along with the presentation.
  \item[-] The examination will revolve around both the worksheets and the paper.
\end{itemize}

\subsection{Paper}
\begin{itemize}
  \item[-] Fill in headlines in the template, what goes where.
\end{itemize}

\subsection{Observer}
\begin{itemize}
  \item[-] A reduced order observer has been designed to determine the angular velocities.
  \item[-] The Vicon system returns very good measurements, so an observer would not be needed, since the derivative of these angles would be sufficient.
  \item[-] It is not necessary when only using vicon, but for the learning perspective, it is nice to implement in the system.
  \item[-] The observer is especially relevant when using on-board sensors like a gyroscope, so designing it now will make the project flexible for later such implementation.
\end{itemize}

\subsection{Miscellaneous}
\begin{itemize}
  \item[-] To get an acceleration of \si{1 m/s^2} we need around \si{0.5 rad} - weather or not this is a large acceleration is difficult to say. It is not ridicules number.
  \item[-] We can compute the maximum acceleration from the mass and thrust force of the quadrotor.
  \item[-] When choosing the magnitude of acceleration we should take the angular disturbances into account.
\end{itemize}

\subsection{Next Supervisor Meeting}
Wednesdays, 12th of October at 10.00

\end{document}