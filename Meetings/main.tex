%implementing document formatting:
\input{preamble.tex}
%Vectors
\renewcommand{\vec}[1]{\boldsymbol{\mathbf{#1}}}
\begin{document}
\renewcommand\chaptername{KAPITEL}
\renewcommand\contentsname{Indhold}
\renewcommand\figurename{Figur}
\renewcommand\tablename{Tabel}

\section*{Supervisor meeting\\ \small Wednesday, 7th of September 2016}

\subsection{Vicon}
\begin{itemize}
  \item[-] Access and introduction for vicon room done.
  \item[-] Vicon system is fast enough for stability control.
  \item[-] The information for stability control must be revived by the controlling unit at a rate of 5-10 Hz for stable flight.
\end{itemize}

\subsection{Hardware}
\begin{itemize}
  \item[-] Control externally and access to internal measurements is possible with the older quad rotor.
  \item[-] New platform is also an option, either educational or hobby.
\end{itemize}

\subsection{Project Scope}
\begin{itemize}
  \item[-] Distributed systems are two separate systems interacting.
  \item[-] On board local stabilization control is an option.
  \item[-] Information can be passed down from the quadrotor (rotor speed, angles, etc).
  \item[-] Significantly easier to control it with a remote than flying between coordinates.
\end{itemize}

\subsection{Meetings \& Reading Material}
\begin{itemize}
  \item[-] Send material 24 hours in advance.
  \item[-] Usually meetings on Wednesdays at 11.00.
  \item[-] The entire report must be sent every time along with pages to be reviewed.
  \item[-] Mail subject should contain group-number.
\end{itemize}

\subsection{Worksheet \& Paper}
\begin{itemize}
  \item[-] The paper should be 6-8 pages.
  \item[-] When writing the paper, assume that people are at a very high level of knowledge.
  \item[-] The worksheets do not have to look nice.
\end{itemize}

\subsection{Miscellaneous}
\begin{itemize}
  \item[-] Reading material: Anders will provide some student reports.
  \item[-] We should think of how far we want to go with modeling.
  \item[-] The coordinate system rotates with the quadrotor.
  \item[-] The "global" coordinate system is called inertial.
  \item[-] The "local" coordinate system is called body.
  \item[-] Z-axis pointing down (up is negative).
  \item[-] The shape of the wing is called airfoil.
  \item[-] Primarily pushing air: Newtonian.
  \item[-] Bernoulli effect is caused by the shape of the airfoil.
  \item[-] Propeller speed is what we need to control and airfoil dynamics can be neglected in near equilibrium flight.
  \item[-] When moving sideways some z-axis-thrust is lost, which means you have to throttle up.
\end{itemize}

\subsection{Next Supervisor Meeting}
Wednesdays, 21st of September at 11.00

\end{document}