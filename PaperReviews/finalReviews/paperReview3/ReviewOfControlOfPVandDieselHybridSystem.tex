%implementing document formatting:
\input{preamble.tex}
%Vectors
\renewcommand{\vec}[1]{\boldsymbol{\mathbf{#1}}}
\begin{document}
\renewcommand\chaptername{KAPITEL}
\renewcommand\contentsname{Indhold}
\renewcommand\figurename{Figur}
\renewcommand\tablename{Tabel}

\section*{Review of Paper on\\
Control of PV and Diesel Hybrid System\\
\small Wednesday, 30th of November 2016}
\subsection{General Assessment}
It is not finished, and therefore an overall impression is hard to draw. However the language and tone is nice.

\subsection{General Comments}
\noindent Figure texts should not state “that it shows, ...” , it should simply state ‘what is in the figure’.
Do not start a sentence with a variable. Then write: the variable, X, is …
Example: The damping factor, $\zeta$, .. \\


\noindent Consistency with spacing between word and source. \\


\noindent  ... on figure should be ... in figure \\


\noindent Do not refer to an equation immediately before it is shown. One way is to write the equation as part of the sentence. \\


\noindent Consistency is missing in relation to math notation. Either use $\cdot$ everywhere or nowhere. We prefer nowhere.  \\


\noindent The numbers in the model equations are not important in general. As an exception it is a good idea to include the numbers of the model which is used for design, such that the core design can be tracked by the reader. \\

\noindent Do not write the actual matlab command - it is for no use for the reader. Leave out the use of matlab \\

\noindent Matrices must be fat ink or with an arrow above to indicate it is matrices.
When mentioned in the text, it shall be in math mode to be shown in italic \\


\noindent Use eqref rather than what you do now for equations. \\



\subsection{Specific Comments}

\noindent Headline information: include your email addresses.
Leave out address of the department. \\


\noindent I-Introduction: Describe a bit more the figure, as the reader does not gain anything from it at this point. Note: Arrows are confusing. \\


\noindent II. Physical System Description: Generally too thorough explained. It will not fit when the rest of the paper is written. Try to be more short and on point \\


\noindent II.A- Physical System Description: $\zeta$ does not decide. The maximum overshoot is given/related by $\zeta$ … \\


\noindent Acknowledgement: Remember to send the final edition of the paper to the persons, to accept the acknowledgements.\\

\noindent References: Comma in the title. You may want to delete it. 
Example: “Control-oriented first principles-based model of a diesel generator,”\\


\noindent “These expectations should be obtained under conditions which at all time ensure the amount of PV power to be as close as possible to the available PV power. Based on this structure PV power will be maximized.” \\
It seems like there are two meanings of PV power in this sentence, maybe it should be written more clearly.


\end{document}