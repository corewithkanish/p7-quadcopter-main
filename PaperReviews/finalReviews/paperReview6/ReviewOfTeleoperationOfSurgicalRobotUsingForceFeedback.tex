%implementing document formatting:
\input{preamble.tex}
\usepackage{enumitem}
\setlist{leftmargin=5.5mm}
%Vectors
\renewcommand{\vec}[1]{\boldsymbol{\mathbf{#1}}}
\begin{document}
\renewcommand\chaptername{KAPITEL}
\renewcommand\contentsname{Indhold}
\renewcommand\figurename{Figur}
\renewcommand\tablename{Tabel}

\section*{Review of Paper on\\
Teleoperation of Surgical Robot using Force Feedback\\
\small Wednesday, 30th of November 2016}
\subsection{General Assessment}
Interesting project and generally well written. Keep it short, focus on what you do in the project.

\subsection{General Comments}
As a reader, we are left with a feeling, that there is a lot of repetition. The introduction can be shortened and be more on point. The content is nice though.\\

\vspace{-7pt}\noindent
Numbers less than 11 must be written with letters.\\

\vspace{-7pt}\noindent
Be consistent with periods/full stop in figure texts.\\

\vspace{-7pt}\noindent
It is "in" figure and not "on" figure.\\

\vspace{-7pt}\noindent
Include sources when theory is stated.\\

\vspace{-7pt}\noindent
Concerning the figures of the given setup and your test setup: We prefer it as functional diagrams rather than pictures (except figure 3). What do you want the reader to gain from the figures?\\

\vspace{-7pt}\noindent
Concerning last paragraph of the discussion: We would prefer it removed and brought in an additional section after the conclusion called "Future Work", as safety has not been in the scope of your project.

\subsection{Specific Comments}
\begin{itemize}
	\item[-] Abstract: \\
           \textbf{Suggestion:} Present the advantages of using robots and MIS before presenting the use of haptic feedback and the challenges of it. 

	\item[-] Section II A - 2nd column - 1st paragraph: "[...] arms with 6 - 7 actuated DOF each." \\
           \textbf{Suggestion:} Write degrees of freedom (DOF) the first time

	\item[-] Section II A: "System Overview" and "A entire setup"
	         There should either be a top for the section, or it should not be followed by a subsection.

	\item[-] Section II: \\
           \textbf{Suggestion:} A lot of space is spend describing the given system. We suggest to shorten it, as maybe you do not want it to be the focus of the paper. 

	\item[-] Section II - 1st column - 3rd paragraph: "[...] (or EndoWrist, more precisely), [...]"\\
           \textbf{Suggestion:} Change the brackets to commas. 

	\item[-] Section III - 1st column - 3rd paragraph: "In this manner, the feedback vector is transformed from Cartesian space to a task space in which the chosen actions form a basis."\\
           \textbf{Thoughts:} We are unsure what you mean by this. Please elaborate with explanation.\\
           \textbf{Suggestion:} We prefer operational space, or a more intuitive explanation of what task space is.
\pagebreak
	\item[-] Section III B, 1st column - 5th paragraph: "A piecewise linear expression is made from the 340 mA sample and up and can be seen on equation(12)". \\
           \textbf{Note:} The approximation is not linear, but affine, since it does not pass through origin.\\
           \textbf{Suggestions:} Maybe include some more explanation here too. For 340 mA, make a reference to figure 6. Figure 4 is presented in subsection B, but is referred to in subsection C. We suggest moving it to subsection C.

	\item[-] Section IV: \\
           \textbf{Thoughts:} Communication should consider delays, noise and computation priority within the microcontroller. Maybe shorten the protocol and write about delays. Last paragraph of the section starts well - what does it conclude. We suggest that you expand the last paragraph and make it the focus of the section. 
\end{itemize}
\end{document}