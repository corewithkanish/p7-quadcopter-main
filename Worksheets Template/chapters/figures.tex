\chapter{Figures}

% Include one figure
\begin{figure}[H] 
	\centering
	\includegraphics[scale=1.3]{figures/CubliCorner-700x4302}
	\caption{A Cubli balancing on one of its corners.\cite{RAndrea}}
	\label{CubliCorner}
\end{figure}\vspace{-18pt}

% Include to pictures in a row, with different captures
\begin{minipage}{\linewidth}
	\begin{minipage}{0.45\linewidth}
		\begin{figure}[H]
			\includegraphics[scale=.5]{figures/PotentiometerResolution}
			\centering
			\captionsetup{justification=centering}
			\captionof{figure}{Potentiometer measurements in volts and the corresponding values that the ADC provides.}
			\label{PotentiometerResolution}
		\end{figure}
	\end{minipage}
	\hspace{0.03\linewidth}
	\begin{minipage}{0.45\linewidth}
		\begin{figure}[H]
			\includegraphics[scale=.5]{figures/PotentiometerResolutionDegRad}
			\centering
			\captionsetup{justification=centering}
			\captionof{figure}{Potentiometer measurements converted to radians and degrees.\vspace{12pt}}
			\label{PotentiometerResolutionRadDeg}
		\end{figure}
	\end{minipage}
\end{minipage}

% Include to pictures in a row, with different captures and a common one
\begin{figure}[H]
	\begin{minipage}{\linewidth}
		\captionsetup[subfigure]{font = footnotesize}
		\centering
		\subcaptionbox
		{
			Here \si{f(x_a) < f(x_b)} resulting in the red interval, \si{x_{L} < x^* < x_a}, and green interval, \si{x_a < x^* < x_b}, which when combined yields the new bracket, \si{[x_{L},\ x_b]}, shown in blue.
			\label{dichotomousLargerB}
		}
		{
			\includegraphics[scale=.6]{dichotomousLargerB}
		}\quad
		\subcaptionbox
		{
			Here \si{f(x_b) < f(x_a)} resulting in the green interval, \si{x_a < x^* < x_b}, and red interval, \si{x_b < x^* < x_{U}}, which when combined yields the new bracket, \si{[x_a,\ x_{U}]}, shown in blue.
			\label{dichotomousLargerA}
		}
		{
			\includegraphics[scale=.6]{dichotomousLargerA}
		}
		\caption{The function, \si{f(x)}, is only evaluated at the points indicated by red dots. Two examples of how the graph of \si{f(x)} could appear is shown in green and red.}
		\label{dichotomousLargerAorB}
	\end{minipage}
\end{figure}\vspace{-18pt}

% Tikz diagrams

\begin{figure}[H]
	\input{figures/blockDiagram.tikz}
	\caption{Block diagram of the Cubli as a SISO system. The input is the torque applied to the wheel. The output is the angular position of the frame.}
	\label{cubliSimulink}
\end{figure}

\begin{figure}[H]
	\input{figures/blockDiagramController.tikz}
	\centering
	\caption{Block diagram of the final controlled system.}
	\label{blockDiagramController}
\end{figure}\vspace{-18pt}